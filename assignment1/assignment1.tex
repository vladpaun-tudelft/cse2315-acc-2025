\documentclass[a4paper,11pt]{article}

% ---------- Packages ----------
\usepackage[margin=2.5cm]{geometry}
\usepackage{amsmath, amssymb, amsthm, mathtools}
\usepackage{algorithm}
\usepackage{algpseudocode}
\usepackage{enumitem}
\usepackage{graphicx}
\usepackage{booktabs}
\usepackage{hyperref}
\usepackage{xcolor}

% ---------- Theorem Environments ----------
\newtheorem{theorem}{Theorem}
\newtheorem{lemma}{Lemma}
\newtheorem{definition}{Definition}
\newtheorem{proposition}{Proposition}
\newtheorem{corollary}{Corollary}

% ---------- Custom Commands ----------
\newcommand{\R}{\mathbb{R}}
\newcommand{\N}{\mathbb{N}}
\newcommand{\Z}{\mathbb{Z}}
\newcommand{\Alphabet}{\Sigma}
\newcommand{\eps}{\varepsilon}
\newcommand{\set}[1]{\left\{ #1 \right\}}
\newcommand{\abs}[1]{\left| #1 \right|}
\newcommand{\ceil}[1]{\left\lceil #1 \right\rceil}
\newcommand{\floor}[1]{\left\lfloor #1 \right\rfloor}

% ---------- Algorithm Style ----------
\algrenewcommand\algorithmicrequire{\textbf{Input:}}
\algrenewcommand\algorithmicensure{\textbf{Output:}}

% ---------- Title Info ----------
\title{\textbf{CSE2315 — Assignment 1}}
\author{Vlad Paun \\ 6152937}
\date{\today}

% ---------- Document ----------
\begin{document}

	\maketitle
	%\thispagestyle{empty}
	%\newpage

	\section{Exercise 1}
	Consider a language $L=\set{ok,a,bad,dab,abba,hi,\eps,acc,duck}$
	\subsection*{a) Give a possible $\Alphabet$ such at $L \subseteq \Alphabet^*$}
	\[
	\Alphabet = \set{a,b,c,d,h,i,k,o,u}
	\]
	\subsection*{b) Why is this only possible $\Alphabet$}
	Supposing this question is asking why this is the only possible \textbf{minimal} alphabet, the answer is that it must contain exactly the set of symbols that appear in the words in $L$, and that set is uniquely determined.
	\subsection*{c) Give all words in $L$ in shortlex order}
	\[
	\eps,a,hi,okacc,bad,dab,abba,duck
	\]
	\section{Exercise 2}
	Consider the following claims (a) and (b). For each claim, verify whether it is true for arbitrary languages $L_1\subseteq\Alphabet^*_1$,$L_2\subseteq\Alphabet^*_2$,$L_3\subseteq\Alphabet^*_3$,$L_4\subseteq\Alphabet^*_4$. If a claim is true, give a proof; if it is not true, give a counterexample with an explanation how the counterexample shows the claim is false
	\section{Exercise 3}
	\section{Exercise 4}
	\section{Exercise 5}
	\section{Bonus Exercise}
	Clearly describe the problem. Use math mode where appropriate:
	\[
	f(n) = \sum_{i=1}^{n} i^2
	\]

	\section{Approach}
	Explain your reasoning, definitions, and structure.

	\subsection{Definitions}
	\begin{definition}
		A function $f : \N \to \N$ is \emph{monotone} if $f(n+1) \ge f(n)$.
	\end{definition}

	\subsection{Lemma Example}
	\begin{lemma}
		For all $n \in \N$, $n^2 \ge n$.
	\end{lemma}

	\begin{proof}
		For $n \ge 1$, we have $n^2 - n = n(n-1) \ge 0$.
	\end{proof}

	\section{Algorithm}
	\begin{algorithm}[H]
		\caption{Example Algorithm}
		\begin{algorithmic}[1]
			\Require Integer $n$
			\Ensure Sum of first $n$ integers
			\State $s \gets 0$
			\For{$i \gets 1$ to $n$}
			\State $s \gets s + i$
			\EndFor
			\State \Return $s$
		\end{algorithmic}
	\end{algorithm}

	\section{Results}
	Tables look clean with \texttt{booktabs}:

	\begin{center}
		\begin{tabular}{ccc}
			\toprule
			$n$ & Value & Time \\
			\midrule
			10 & 55 & 0.01s \\
			100 & 5050 & 0.02s \\
			\bottomrule
		\end{tabular}
	\end{center}

	\section{Conclusion}
	Summarize findings briefly.

\end{document}